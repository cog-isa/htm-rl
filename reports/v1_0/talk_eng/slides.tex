\documentclass[default]{beamer}
\usepackage{slides_style}
\setbeamertemplate{navigation symbols}{}
\usetheme{Boadilla}

\begin{document}

\title[]{
	Planning with Hierarchical Temporal Memory for Deterministic Markov Decision Problem
}
\author[Petr Kuderov]{
    Petr Kuderov, Aleksandr I. Panov
}
\institute[MIPT]{
    Moscow Institute of Physics and Technology, Moscow, Russia\\
	kuderov.pv@phystech.edu
}
\date[]{Feb 2021}
	
\begin{frame}
	\titlepage
	\centering
	\includegraphics[height=20pt]{mipt_logo} \hspace{10pt}
	\includegraphics[height=20pt]{cdsl_logo} \hspace{10pt}
\end{frame}

\begin{frame}{Introduction and motivation}
	\begin{itemize}
		\item why we need a model
		\item why we draw inspirations from human memory models
	\end{itemize}
\end{frame}

\begin{frame}{Research question}
	- Check applicability of TM
  - Build planning method
\end{frame}

\begin{frame}{HTM and TM}
	SDR, prediction and backward prediction	
\end{frame}

\begin{frame}{Agent's architecture}
	- figure
	- explain how it works on a high level	
\end{frame}

\begin{frame}{Planning, stages}
	- the goal of planning
	- two stages, what their goals
	- forward planning: goal reachability and keep track of predictions
	- backtracking: find the exact path to the goal
\end{frame}

\begin{frame}{Forward planning}
	explain how it's done
\end{frame}

\begin{frame}{Backtracking}
	explain recursive algo
\end{frame}

\begin{frame}{Results. Experimental setup 1}
	- explain setup
	- figures
	- results	
\end{frame}

\begin{frame}{Results. Experimental setup 2}
	- explain setup
	- figures
	- results	
\end{frame}

\begin{frame}{Conclusions}
	regarding research questions
\end{frame}

\begin{frame}{Limitations and possible improvements}
	- deal with planning horizon: two ways
	- regading random strategy	
\end{frame}

\end{document}