\documentclass[a4paper]{article}
\usepackage{style}

\begin{document}

\title {Применение Temporal Memory фреймворка HTM в задачах обучения с подкреплением}
\author {Petr Kuderov}
\maketitle

\begin{abstract}
  TBD
\end{abstract}

\section{Введение}

Скелет
\begin{enumerate}
  \item abstract
  \item introduction
  \item problem in general
  \item related works
  \item problem setting and terminology: RL, HTM
  \item method
  \item testing
  \item results and outro
\end{enumerate}

Текущие вопросы
\begin{itemize}
  \item Цель: проверить применимость использования памяти агента на основе TM для моделирования среды (функции перехода)?
  \item Сужать постановку задачи до model based RL?
  \item Тестирование проводилось как для обычной задачи RL с неизменной наградой, так и для задачи meta-RL. Упоминать ли второе?
  \item Является ли метод планирования частью изначальной цели? Ведь именно ему посвящена статья.
\end{itemize}

\section{Related works}

\section{Постановка задачи}

Рассматривается 

\section{Описание метода}

\section{Результаты тестирования}

\section{Выводы}

\end{document}
