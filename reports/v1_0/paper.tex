\documentclass[a4paper]{article}
\usepackage{style}

\begin{document}

\title {Применение Temporal Memory фреймворка HTM в задачах обучения с подкреплением}
\author {Petr Kuderov}
\maketitle

\begin{abstract}
  TBD
\end{abstract}

\section{Введение}

Eng title in easychair: An example of biologically inspired memory-based method applied to Reinforcement Learning problem setting.

Eng abstract:
Hierarchical Temporary Memory (HTM) is a biologically inspired model of neocortex. Its key properties proved it to be useful useful for sequence learning and anomaly detection problems. While they also make it potentially useful for a model-based RL, this particular application of HTM has hardly been studied yet.

We show a possible way to utilize HTM in RL setting to model an environment dynamics. We accompany it with an example of planning algorithm which is used to test learning abilities and find limitations of such a model.

Скелет
\begin{enumerate}
  \item abstract
  \item introduction
  \item problem in general
  \item related works
  \item problem setting and terminology: RL, HTM
  \item method
  \item testing
  \item results and outro
\end{enumerate}

Текущие вопросы
\begin{itemize}
  \item Цель: проверить применимость использования памяти агента на основе TM для моделирования среды (функции перехода)?
  \item Сужать постановку задачи до model based RL?
  \item Тестирование проводилось как для обычной задачи RL с неизменной наградой, так и для задачи meta-RL. Упоминать ли второе?
  \item Является ли метод планирования частью изначальной цели? Ведь именно ему посвящена статья.
\end{itemize}

\section{Related works}

\section{Постановка задачи}

Рассматривается 

\section{Описание метода}

\section{Результаты тестирования}

\section{Выводы}

\end{document}
