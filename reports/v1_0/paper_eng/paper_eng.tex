\documentclass[runningheads]{llncs}
\usepackage{style_eng}
\bibliographystyle{splncs04}

\begin{document}

\title {Planning with Hierarchical Temporal Memory for deterministic Markov Decision Problem}
\author {Petr Kuderov \inst{1} \and Aleksandr I. Panov \inst{1}}

\authorrunning{P. Kuderov, A. Panov}
% First names are abbreviated in the running head.
% If there are more than two authors, 'et al.' is used.
%
\institute{Moscow Institute of Physics and Technology, Russia}

\maketitle

\begin{abstract}
  Among the key problems in artificial intelligence is sequential decision making, which can be formalized as Markov Decision Process (MDP). One approach to solve it, called model-based Reinforcement Learning (RL), combines learning the model of the environment and the global policy. Having a good model of the environment opens up such properties as data efficiency and targeted exploration.
  
  While most of the memory-based approaches are based on using Artificial Neural Networks (ANNs), in our work we instead draw the ideas from Hierarchical Temporary Memory (HTM) framework, which is based on human-like memory model. We utilize it to build an agent's memory that learns the environment dynamics. We also accompany it with an example of planning algorithm, that enables the agent to solve RL tasks.
\end{abstract}

\keywords{Model-based reinforcement learning \and Markov Decision Process \and planning \and Hierarchical Temporal Memory \and Sparse Distributed Representation.}

\section{Introduction}

Reinforcement Learning (RL) seek to find methods that enable robots or virtual agents to act autonomously in order to solve human-defined tasks. It requires from such methods to learn a policy for sequential decision-making problems, which are commonly formalized as MDP optimization \cite{Puterman_1994}. Recently RL has achieved notable landmarks in various task domains such as classical board games (chess \cite{Campbell_Hoane_Hsu_2002_DeepBlue}, Go \cite{Silver_Go}, poker \cite{Brown_Sandholm_2017_Poker}) and visually rich computer games (Atari \cite{Mnih_2015_Atari}, Dota 2 \cite{openai_2019_dota}). At the core of some of the methods used, especially for board games, were planning algorithms, which rely on the knowledge of the environment dynamics.

In general case, though, the model of the environment is unknown, and there are two common opposite approaches in RL to deal with it: a model-free and a model-based. The former approach doesn't explicitly learn and make use of the knowledge of the environment. Instead, it directly learns the global policy from interactions with the environment. While model-free approach has many successful examples (\cite{Mnih_2015_Atari,haarnoja_2018_sac,schulman_2017_ppo}), among its disadvantages are data inefficiency, as it requires high amount of interactions with the environment, and struggle for precise lookahead.

Model-based RL can address issues of both planning and model-free RL by first learning the environment dynamics, and then to exploit it for more efficient learning of the agent's policy (\cite{sutton_barto_2018}). Typically, the model is represented by MDP and consists of two components: a state transition model and a reward model. After the model has been learned, MDP planning method such as MCTS (\cite{Coulom_2007_mcts}) can be applied to infer the optimal policy. Many model-based RL approaches build a model that operates at the pixel-level (\cite{kaiser_2020_modelbased}), which is hungry for computational resources and attentive to unimportant details. Another approach is to try building a latent-space model. For example, a latent-space model can represent an abstract MDP that's equivalent to the real one in terms of state values, i.e. the cumulative reward of the planned trajectory (\cite{silver_2020_muzero}).

To model the environment it's common to use ANNs (\cite{silver_2020_muzero,Ha_Schmidhuber_2018_worldmodels}). It's also very tempting to take more inspiration from human brains, especially for how our memory functions (\cite{Hassabis_2017_neuro}). One example of such human-like memory model is Hierarchical Temporal Memory (HTM) framework (\cite{George_Hawkins_2009}). This framework utilizes discrete binary sparse distributed representations (SDR) (\cite{ahmad_2016_sdr}), which enables more efficient data processing, high noise tolerance and a natural way to set similarity metric between data embeddings. It also provides a model of the sequential temporal memory (TM) (\cite{hawkins_TM}). Learning in TM isn't based on the gradient descend, and typically it learns with a faster rate compared to common ANNs.

In our work we present a simplified memory-based framework based on the ideas of HTM that can be used to model the environment dynamics, supplied with the planning algorithm to infer the optimal global policy. We study applicability of the presented memory model and viability of the planning method in different maze environments.

\begin{figure}
  \centering
  \begin{minipage}{.49\linewidth}
    \includesvg[width=\linewidth]{gridworld_5x5_1_1_200_1337_map_0_1173222464}
  \end{minipage}
  \begin{minipage}{.49\linewidth}
    \includesvg[width=\linewidth]{gridworld_5x5_1_1_200_1337_map_2_1561234712}
  \end{minipage}
  \caption{Example of generated maze environments of 5x5 and 8x8 sizes. Walls - dark purple, starting agent's position - yellow, rewarding goal state - salad.} \label{fig_maze_examples}
\end{figure}

\section{Background}

\subsection{Reinforcement Learning}

In this paper we consider the classic decision-making problem with an agent operating in an environment, which is formalized as Markov Decision Process (MDP). As a starting point in the research we simplify the problem assuming that the environment is fully observable and deterministic. We also consider only environments with the distinctive goal-oriented tasks, where an agent is expected to reach desirable goal states in a limited time.

Therefore, MDP is defined as a tuple $M = (\St, \A, R, P, s_0, s_g)$, where $\St = \{1, \dots, s_n\}$ is the state space of the environment, $\A = \{1, \dots, a_m\}$ is the action space available to an agent, and $s_0$ and $s_g$ are the initial and the goal state distributions respectively. At each time step $t$, an agent observes a state $s_t \in \St$ and selects an action $a_t \in \A$. As a result, it receives the reward $r_t \in R(s_t, a_t)$ and transitions to the new state $s_{t+1} = P(s_t, a_t)$. An agent's policy $\pi: S \rightarrow A$ is a mapping from a state to an action. The agent's goal is to maximize the expected return $\E \left[ \sum_{t=0}^{\infty} r_t \right]$. 

We don't use discount factor $\gamma$ in the return calculation. Instead, we assume that the reward function is designed the way that it highly rewards reaching the goal states and slightly punishes in any other states, which should reflect the goal-oriented nature of the environments that were taken into consideration.

\subsection{Temporal Memory}

Temporal Memory (TM) from Hierarchical Temporal Memory (HTM) framework is a model of the sequential memory. It works with sparse distributed representations (SDR), which are sparse binary vectors, typically high-dimensional. Therefore, the data encoding scheme should be chosen too.

In our work both actions and states are represented as integer numbers from a fixed range (although, different for states and actions). Thus, we use a simple encoding scheme, where numbers from some range $[0, N)$ are mapped to non-overlapping equally-sized sets of active bits of the output SDR vector. For example for $N = 3$ numbers 0, 1 and 2 are encoded correspondingly as following:

$$
\begin{aligned}
  0 & \rightarrow \verb|1111 0000 0000| \\
  1 & \rightarrow \verb|0000 1111 0000| \\
  2 & \rightarrow \verb|0000 0000 1111| \\
\end{aligned}
$$

As you see, resulting vector is split into buckets of bits, and the size of buckets is a hyperparameter (in the example above it's 4).

For SDR vectors a union operation is defined as a bitwise OR applied to the corresponding vectors. Also we define a similarity metric, which is the number of the same one-bits of the vectors in consideration.

Temporal Memory model not only provides an algorithm to learn sequences of SDR vectors $[v_0, v_1, v_2, \dots ]$ but also to make predictions about what SDR vector $v'$ comes next based on the past experience. The prediction is represented also as an SDR vector, which is a union of SDR vectors $v' = \cup v_{o_i}$, where each vector $v_{o_i}$ represents a single separate expected outcome. Notice that as the union of SDR vectors is also an SDR vector, TM model allows you to work simultaneously with the unions of sequences as well.

\section{Planning with Temporal Memory}

In our method we train an agent to reach rewarding goal states $s_g$. An agent is supplied with the memory module to learn the environment's dynamics and a planning algorithm which heavily relies on the learned model. The planning algorithm constructs a plan of actions $p = [a_t, a_{t+1}, \dots, a_{t+h-1}]$ that leads to the desired state $s_g = s_{t+h}$ from the current state $s_t$. In the case when the planning algorithm fails to provide a plan we switch the agent to the exploration strategy $\pi_r$ in order to improve the learned model. As the result, at each timestep an agent operates either according to a constructed plan $p$ or according to the random uniformly-distributed exploration policy $\pi_r$.

\subsection{Memory module}

The agent's memory module consists of two submodules. One submodule, called \textit{transition memory}, learns the transition model $f: \St \times \A \rightarrow \St$ of the environment, while the other, called \textit{goal states memory} keeps track of the rewarding goal states $s_g$.

Consider an agent's trajectory $\tau = [s_0, a_0, r_1, s_1, a_1, r_2, \dots, r_{T-1}, s_T]$. Each pair $(s_t, a_t)$ at the moment $t$ defines the precondition to the transition in the environment $(s_t, a_t) \rightarrow s_{t+1}$. Therefore, we made transition memory to learn the sequences of such transitions $Tr = [(s_0, a_0), (s_1, a_1), \dots]$ from the trajectories that the agent produces. The learning is performed in an online fashion, i.e. at each timestep.

The transition memory submodule is based on a Temporal Memory.Recall that TM works with sparse distributed representations (SDR). To encode each state-action pair $(s, a)$ we use two separate integer encoders for states and actions respectively to obtain binary sparse vectors. The states encoder maps integer numbers from $[0, |\St|)$ range, and the actions encoder maps integer numbers from $[0, |\A|)$ range. As most of the time our method works with SDR, for simplicity further on we use the same notation $s$ and $a$ for their corresponding SDR too. To work with pairs $(s, a)$ we just concatenate their SDR vectors together into one.

For any given state-action pair $(s, a)$ the transition memory module can give a prediction on what state $s'$ is coming next.
Precisely, it allows prediction queries of the form:

\begin{quote}
  \textit{given a superposition \footnotemark of states $S = \cup s_i$ and a superposition of actions $A = \cup a_j$, what superposition of next states $S'= \cup s'_m$ all possible state-action pairs $(s_i, a_j)$ lead to?}
\end{quote}

\footnotetext{TM can work with unions of SDR vectors. When SDR vectors in a union are semantically meaningful, e.g. when they represent states or actions, we call such unions \textit{superpositions} of states or actions respectively \footnotemark.}

\footnotetext{As you see, in our current encoding scheme the different states or actions are encoded with non-overlapping bits which makes them distinguishable from each other in a superposition. Whilst it's true, the current planning algorithm only requires a separation of the state bits from the action bits in a state-action pair SDR. This is done intentionally. Non-overlapping scheme makes things simpler to read and debug, but cannot keep the similarity between states (or observations). For the future work we plan to experiment with more complex environments and therefore to switch to overlapping mappings.}

In some specific circumstances the transition memory submodule can be used backward - as a reverse of an environment transition function $f$. This feature becomes available after getting the result to a prediction query and only for the states $s'_m$ that match the result $S' = \cup s'_m$. For each such state $s'$ it can provide which state-action pairs from the original query lead to $s'$: $\{(s, a)\} = f^{-1}(s')$. Notice that in general case it's possible that \textit{a multiple} state-action pairs lead to $s'$.

\subsection{Planning algorithm}

Planning process is split into two stages (see function ~\ref{func_PlanActions}):

\begin{enumerate}
  \item Forward planning from the current state until reward is reached.
  \item Deducing the backward path from the reward state to the current.
\end{enumerate}

\begin{function}
  \caption{PlanActions($s_0$)}\label{func_PlanActions}
  \SetAlgoNoLine
  $s_0$: initial state \;
  $g: $ reached goal \;
  $H: $ Temporal Memory active segments history \;
  \quad \\

  $g, H \leftarrow PredictToGoal(s_0)$ \;
  $plan \leftarrow BacktrackFromGoal(g, s_0, H)$ \;

  \Return{$[a_0, a_1, .. a_T] = plan$}
\end{function}

\subsubsection{Forward planning}

Agent uses its experience and expects the reward to be in one of the previous rewarding states. But even if it is wrong, we expect that direct goal-based exploration is more optimal than random walking. Hence it keeps the goal list and gives such goal states a priority.

Consider that at some timestep $t_0$ the agent is in state $s_0$. To find out if there is a reachable goal state in a radius of $N$ actions, it start planning from the superposition $(S_0 = s_0, A = \cup a^i)$, where $A$ is a union of all available actions (we use superscripts for indices here as subscripts are used to denote timestep). It is similar to making all possible actions simultaneously in parallel. To achieve this planner uses the memory and gets a superposition of next state-action pairs.

After the first planning step the planner has a superposition of the next states $S_1 = \cup s^i$, reachable in one agent's action. To get all reachable states in two agent's action, planner makes another prediction step - now from the superposition $(S_1, A)$, and so on (see function ~\ref{func_PredictToGoal}).

\begin{function}
  \caption{PredictToGoal($s_0$)} \label{func_PredictToGoal}
  \SetNoFillComment
  \KwData{$s_0$ - initial state}

  $sa_0 \leftarrow Encode((s_0, \cup a^i))$ \tcp*{Init starting superposition}
  $H \leftarrow []$ \tcp*{Init active segments history}

  \ForEach {timestep $i$} {
    $memory.ActivatePredict(sa_i)$ \;
    \quad \\

    $H[i] \leftarrow memory.active\_segments$ \;
    $sa_{i+1} \leftarrow PredictionFromSegments(H[i])$ \;
    \quad \\
    
    $s, a \leftarrow Decode(sa_{i+1})$ \tcp*{Both $s$ and $a$ are superpositions!}
    $reached\_goal \leftarrow MatchGoals(s)$ \;
    \If(){$reached\_goal \neq \emptyset$} {
      \Return {$reached\_goal, H$} \;
    }{}
  }
  \Return{$\emptyset$}
\end{function}

This process is continued until either some state from the goal list is reached, or planner exceeds its planning horizon limit. In the former case planner proceeds to the next state of planning process. In the latter, agent uses its random strategy to make a move before it tries to plan again. To test if the goal state is in superposition we intersect their sparse vectors and sum active bits, which is the same as dot product of such vectors. If they match is over some predefined threshold, the state is considered presented in this superposition.

During this stage planner collects a history of active superpositions $H = [S_0, S_1, \dots S_T]$. In more details, he keeps the track of Temporal Memory cells' segments activations. Each segment is a pair $(p, \{c_i\})$, where $p$ - index of predicted cell and $\{c_i\}$ is a set of cell indices that induce prediction of $p$. This information allows planner to use inversed environment transition function to get which $\{(s, a)\}$ has lead to a state $s'$ by tracking which cells predict $s'$ cells.

It's possible to find multiple reachable goals at the end of this stage. In our current implementation we use the first matched from the goal list.

\subsubsection{Backtracking}

After the first stage planner ``believes'' that some goal state $g = s_T$ can be reached in $T$ steps. The problem is that he doesn't know how to get there as until now he worked with superpositions. So the main goal of this stage is to restore the sequence of actions leading from $s_0$ to $s_T$. Further on this sequence of action is meant to be the agent's plan in environment.

Planner solves the problem from the end. It moves backward in the history of segments activations graph, trying to get to the starting state $s_0$ at the first timestep. Due to its ``backward moving'' nature we call this stage \textit{backtracking} (see function ~\ref{func_BacktrackFromGoal}).

\begin{function}
  \caption{BacktrackFromGoal($g$, $H$)} \label{func_BacktrackFromGoal}
  % \SetAlgoNoLine
  \KwData{$g = s_T$ - goal state; $H$ - active segments history}

  $T \leftarrow |H| + 1$ \tcp*{As the last recorded transition is for $T-1 \rightarrow T$}
  $S_T \leftarrow PredictionFromSegments(H[T-1])$ \tcp*{State superposition at $T$}
  $s_T \leftarrow s_T \cap S_T$ \tcp*{Keeps only matched bits}
  \quad \\
  $is\_successful, actions \leftarrow Backtrack(s_T, T-1, H)$ \;
  \Return {$is\_successful, actions$}
\end{function}

Recall that planner keeps track of all state superpositions $H = [S_0, S_1, \dots S_T]$ it has reached on each step $t$, and for any state $s'$ from each of these superpositions $S_i$ the memory can answer which state-action pairs has lead to it.

Therefore, given the reached goal state $g = s_T$ planner can get which state-action pair $(s_{T-1}, a_{T-1})$ has lead to it. Hence he knows for $T-1$ step which action is required to get from $s_{T-1}$ to the goal state $s_T$. And also he can recursively reduce the $s_0 \leadsto s_T$ pathfinding problem to the  $s_0 \leadsto s_{T-1}$. Planner repeats this procedure and for each timestep $t$ sequentially finds state-action pairs $(s_{t-1}, a_{t-1}) = f^{-1}(s_t)$ until he returns to the initial state at the first timestep (see ~\ref{func_Backtrack} function).

In general case $f^{-1}(s_{t+1})$ gives multiple pairs $(s_t, a_t)$, and each leads to $s_{t+1}$. But not every pair is actually reachable in exactly $t$ steps from the $s_0$, i.e. not from every pair you can backtrack to the initial state at time $0$. Hence planner sequentially tests each pair until he find the first successful.

As the result of the backtracking stage planner forms a plan of actions $plan = [a_0, a_1, \dots a_{T-1}]$ which he ``believes'' [in terms of his approximation of the environment dynamics] will lead the agent from the initial state $s_0$ to the target state $g = s_T$.

\begin{function}
  \caption{Backtrack($p_t$, $t$, $H$)} \label{func_Backtrack}
  \KwData{$p_t$ - indices of required predicted cells at time $t$, $t$ - current timestep, $H$ - history of active segments}
  
  \If(){$t \leq 0$} {
    \Return{True, $[]$}
  }{}

  $P \leftarrow H[t][p_t]$ \tcp*{List of cell sets, each predicts one of $p_t$ cells}
  $C \leftarrow P$ \tcp*{Initializes cell clusters - candidates for backtracking}
  \quad \\

  \tcp{Iteratively merges any two sufficiently matching cell clusters}
  \While{$\exists c_i, c_j \in C: match(c_i, c_j) \geq \Theta$} {
    $C[c_i] = c_i \cup c_j$ \;
    $C[c_j].remove()$ \;
  }
  \quad \\

  \ForEach{$c \in C$}{
    \tcp{Filters by the amount of $p_t$ cells cluster $c$ predicts}
    $p = PredictionFromCluster(c, H[t])$ \;
    \If(){$match(p, p_t) \le \Omega$}{
      \textbf{continue}
    }{}
    \quad \\
    
    \tcp{Recursively checks candidate cluster $c$}
    $s_t, a_t \leftarrow Decode(ColumnsFromCells(c))$ \;
    
    $is\_successful, actions \leftarrow Backtrack(s_t, t-1, H)$ \;
    \If() {is\_successful} {
      $actions.append(a_t)$ \;
      \Return{True, actions}
    }
  }

  \Return{False, $[]$}
\end{function}

After successful backtracking the planner has a policy for the next $T$ steps, which an agent uses. And if it ends up in a rewarding state, an episode finishes. Otherwise it continues and the planner removes that state from its goal list for the rest of the episode.

\section{Experiments}

In our experiments we decided to use classic GridWorld environments, which are represented as mazes on a square grid (see Figure ~\ref{fig_maze_examples}). Each state can be defined with the position of an agent, i.e. in which cell it stands. Thus, state space $\St$ consists of all possible agent's positions. We enumerate all positions (and therefore states) with integer numbers. An agent starts from the fixed state $s_0$, and its goal is to find a single reward placed at some fixed position $s_g$. Each environments has deterministic transition function $P: \St \times \A \rightarrow \St$. The action space consists of 4 actions $\A = \{ 0, 1, 2, 3 \}$ defining agent's move directions: 0 - east, 1 - north, 2 - west, 3 - south. Each action moves an agent to the adjacent grid cell in the corresponding direction, except when the agent tries to move into the maze wall - in this case it stays in the current cell.

When an agent reaches the goal state $s_g$ it is rewarded with high positive value. Any other steps it is slightly punished. In our experiments we used $+1.0$ for the goal reward and $-0.01$ for the punishment. In each testing environment the optimal path from the starting point to the goal was less than 20 steps. Such reward function makes possible easy episode outcomes differentiation, depending on whether or not the goal state has been reached during it, and if the goal has been reached then how fast.

All experiments were divided into two groups. Each experiment from the first group was held in a fixed environment, while each experiment from the second group had an environment being changed every $N_{ep}$ episodes.

In our experiments we tested ability to model the environment and utilize learned model to plan actions that lead to the goal states. We compared our method's performance with the baselines: random strategy and DQN (\cite{Mnih_2015_Atari}). We also compared performance for the different planning horizons and the number of simultaneously tracked goals in the goal tracking memory.

\subsection{Experiments in fixed environments}

For this experiment setup we used a number of handcrafted mazes. Each experiment had a fixed maze and the starting point $s_0$, but the place of the reward was being changed every $N$ episodes. We tested an agent's ability to find the goal state and how it adapts to the new goal position.

We present results for two mazes: \texttt{multi\_way\_v0} and \texttt{multi\_way\_v2} (see Figure ~\ref{fig_comparison_mw} and ~\ref{fig_detailed_mw0}). Each experiment in \texttt{multi\_way\_v0} had 4 different rewarding goal positions being sequentially changed every 100 episodes. Each experiment in \texttt{multi\_way\_v2} had 6 rewarding goal positions being changed every 200 episodes. Schematic view of these mazes is provided in Appendix ~\ref{apx_handcrafted_mazes}.

\begin{figure}
  \centering
  \begin{minipage}{.49\linewidth}
    \includesvg[width=\linewidth]{multi_way_v0__steps}
    \subcaption{\texttt{multi\_way\_v0}}
  \end{minipage}
  \begin{minipage}{.49\linewidth}
    \includesvg[width=\linewidth]{multi_way_v2__steps}
    \subcaption{\texttt{multi\_way\_v2}}
  \end{minipage}
  \caption{Comparison of the episode duration (timesteps) for the handcrafted mazes. \texttt{htm\_X\_Yg} is our agent with the planning horizon X and Y tracked last goals, \texttt{htm\_0} - random policy, \texttt{dqn\_greedy} - greedy DQN agent. Results are plotted with the moving average MA = 20 steps.} \label{fig_comparison_mw}
\end{figure}

\begin{figure}
  \centering
  \begin{minipage}{.49\linewidth}
    \includesvg[width=\linewidth]{multi_way_v0__dqn__steps}
    \subcaption{Greedy and $\epsilon$-greedy DQN agent}
  \end{minipage}
  \begin{minipage}{.49\linewidth}
    \includesvg[width=\linewidth]{multi_way_v0__1__steps}
    \subcaption{Planning horizon H = 1}
  \end{minipage}
  
  \begin{minipage}{.49\linewidth}
    \includesvg[width=\linewidth]{multi_way_v0__2__steps}
    \subcaption{Planning horizon H = 2}
  \end{minipage}
  \begin{minipage}{.49\linewidth}
    \includesvg[width=\linewidth]{multi_way_v0__4__steps}
    \subcaption{Planning horizon H = 4}
  \end{minipage}

  \caption{Detailed results for \texttt{multi\_way\_v0}. a)Comparison with DQN: blue - greedy policy, yellow - $\epsilon$-greedy policy, green - our method with planning horizon 4 and the last 1 goal tracked; b-d) comparison between tracking all goal states and only the last goal for the different planning horizon. Results are plotted with the moving average MA = 20 steps.} \label{fig_detailed_mw0}
\end{figure}

In the tested environments our method performed better than random policy. But the more complicated an environment was the less distinctive was the difference between the $H = 0$ and $H = 1$. Our method performs better with the increase of the planning horizon $H$. Each increment of the planning horizon is cumulative, i.e. each next increment is more advantageous.

Regarding the goal tracking memory size, we found that increase of the size leads to better exploration. We think that's because an agent less frequently uses random strategy and therefore moves less chaotically. However, this behavior may be suboptimal. For example when the rewarding goal state is reachable from the current position in less than $H$ actions but there's a fake goal state in the opposite direction that is closer to the agent. We think that these results are highly biased due to the special type of handcrafted mazes and initial conditions.
  
We also found that an agent learns and adapts faster than DQN agent (in terms of the number of episodes required to steadily reach the goal during the episode). And longer planning horizon leads to a better adaptation to the changed rewarding position.

When the planning horizon is big enough to reach the goal state from the starting position, then the agent learns the optimal policy very fast. According to our experiments, in this case it learns $\sim$1.5-2.5 times faster than DQN. But for any fixed planning horizon if we start rising an environment complexity, then at some point DQN starts to perform better than our method. That's because DQN always finds an optimal path, and our agent finds it only when it has enough planning horizon.

\subsection{Experiments in changing environments}

In this experiment setup we used a number of randomly generated GridWorld mazes of fixed size (examples provided in Figure ~\ref{fig_maze_examples}). Each experiment has the following scheme. It has been made sequentially in $N_{env}$ randomly generated mazes. For a fixed maze the rewarding goal position was changed sequentially $N_{rew}$ times. For a fixed rewarding goal position the initial position was changed sequentially $N_{s_0}$ times. And for a whole configuration being fixed agent played $N$ episodes. So, each experiment had $N_{env} \cdot N_{rew} \cdot N_{s_0} \cdot N$ total episodes. We tested an agent's ability not only to find the rewarding goal and adapt to its new position, but also how the agent adapts to the whole new maze.

We tested agent in mazes generated on $5 \times 5$, $6 \times 6$ and $8 \times 8$ square grids. In this section we present only the results for the $5 \times 5$ mazes, but for the other tested sizes the results are similar. The key difference in complexity of each experimental setup is the total number of the episodes with fixed reward and maze. The less episodes has agent to adapt, the more agile it's required to be.

We present results for the following experimental setups (Figure 4):

\begin{enumerate}
  \item Experimental setup with the \textbf{rare} change of reward and maze: $N_{s_0} = 100$, $N_{rew} = 2$, $N_{env} = 8$.

  \item Experimental setup with the \textbf{frequent} change of reward and maze: $N_{s_0} = 20$, $N_{rew} = 1$, $N_{env} = 20$.
\end{enumerate}

\begin{figure}
  \centering
  \begin{minipage}{.49\linewidth}
    \includesvg[width=\linewidth]{gridworld_5x5_100_2_8_1337__steps}
    \subcaption{Comparison with baselines}
    \includesvg[width=\linewidth]{gridworld_5x5_100_2_8_1337__dqn__steps}
    \subcaption{Greedy and $\epsilon$-greedy DQN agent}
    \includesvg[width=\linewidth]{gridworld_5x5_100_2_8_1337__2__steps}
    \subcaption{Planning horizon: 2}
    \includesvg[width=\linewidth]{gridworld_5x5_100_2_8_1337__4-8__steps}
    \subcaption{Planning horizon: 4-8}
  \end{minipage}
  \begin{minipage}{.49\linewidth}
    \includesvg[width=\linewidth]{gridworld_5x5_20_1_20_1337__steps}
    \subcaption{Comparison with baselines}
    \includesvg[width=\linewidth]{gridworld_5x5_20_1_20_1337__dqn__steps}
    \subcaption{Greedy and $\epsilon$-greedy DQN agent}
    \includesvg[width=\linewidth]{gridworld_5x5_20_1_20_1337__2__steps}
    \subcaption{Planning horizon: 2}
    \includesvg[width=\linewidth]{gridworld_5x5_20_1_20_1337__4-8__steps}
    \subcaption{Planning horizon: 4-8}
  \end{minipage}

  \caption{Results for experiments in changing environments. Left row: results for the setup with rare changes; right row: results for the setup with frequent changes.}
\end{figure}

From this set of experiments we conclude that in average our agent adapts to changes faster than DQN. Which also makes it perform more stable.

In contradiction to our results for the experiments in the fixed handcrafted environments we found that the increase of the number of the tracked goals makes our method to perform worse.

\section{Conclusion}

We introduced a novel memory-based method that combines learning the model of the environment with the planning algorithm to infer a local policy. We showed that under certain conditions (enough planning horizon) the resulting global policy is optimal or at least very close to it. We also showed that in such cases, compared to DQN, our method learns with the comparable speed or slightly faster. The planning algorithm with any non-zero planning horizon performs better than random strategy. However, for small planning horizons and hard tasks results are sufficiently degraded and not that far from the pure random strategy.

To model the environment dynamics we used human-like model of the memory called HTM. We showed that it's capable to quickly and reliably learn the transition model of the visited states in just a few episodes. As far as we know it is the first time HTM model was used for the model-based RL method. So, even though we applied it to model very simple deterministic environments with perfectly distinguishable states the results are promising, and we look forward to adapt our method to more complex environments.

\bibliography{citations}

\appendix

\section{Appendix}
\subsection{Handcrafted mazes} \label{apx_handcrafted_mazes}

The notation we use in the schematic views of the environments: \verb|-| - empty cell, \verb|#| - wall, \verb|@| - agent, \verb|X| - reward.

\subsubsection{multi\_way\_v0}

Each experiment had 4 different rewarding places being sequentially changed every 100 episodes:

\begin{verbatim}
  #####   #####   #####   #####
  #@--#   #@-X#   #@--#   #@--#
  #-#-#   #-#-#   #-#-#   #-#X#
  #--X#   #---#   #X--#   #---#
  #####   #####   #####   #####
\end{verbatim}

\subsubsection{multi\_way\_v2}

Each experiment had 6 rewarding places being changed every 200 episodes:

\begin{verbatim}
  #########  #########  #########  #########  #########  #########
  #---#####  #-X-#####  #---#####  #---#####  #---#####  #---#####
  #-#-#####  #-#-#####  #-#-#####  #-#-#####  #-#-#####  #-#-#####
  #--@---##  #--@---##  #--@--X##  #--@---##  #X-@---##  #--@---##
  ###--####  ###--####  ###--####  ###--####  ###--####  ###--####
  ###-#--X#  ###-#---#  ###-#---#  ###-#---#  ###-#---#  ###-#--X#
  ###---###  ###---###  ###---###  ###---###  ###---###  ###---###
  ###-#####  ###-#####  ###-#####  ###X#####  ###-#####  ###-#####
  #########  #########  #########  #########  #########  #########
\end{verbatim}

\end{document}
